
\section{Heat integration}
% (4p) François + Ivan
Integration of heat in industrial processes has been practiced historically based on engineering intuition and improvements on existing processes. Linnhoff proposed a systematic methodology for heat integration in 1983 and referred to the technique as 'Pinch analysis'. This development formed the basis of heat integration techniques in industry since, leading to improved efficiency in the design of modern processes and also providing a basis for targeting of retrofit projects in existing industrial sites. The pinch analysis principles are based on separating a process or site into the streams which require heat, cold streams, and those which have excess heat or requires cooling, hot streams. The aggregate hot and cold streams can be represented on a plot of temperature against heat load for the stream to show the total cooling and heating needs of a process and are named composite curves as they represent the total heating and cooling needs of all streams within a process. Designing an integrated and heat-efficient process suggests that the hot process streams and cold process streams can exchange heat directly, or through an intermediate fluid, in order to maximise the recovery of heat within the process and therefore reduce the external utility load. This can be visualised graphically by adjusting the horizontal positioning of the composite curves until they are separated by a minimum temperature (vertical line) of $\Delta T_{min}$. This separation expresses the minimum temperature that is required to ensure a sufficient driving force for heat to flow from the hot to cold stream and is determined by an optimisation accounting for the capital cost of equipment for heat exchange and the cost of providing an external utility to accomplish the thermodynamic requirement. The formulation of the $\Delta T_{min}$ problem is a mixed-integer linear programming problem and can be formulated as shown in Equations \ref{eq:DTminobj} - \ref{eq:DTminfin}.

The first equation in determining the appropriate $\Delta T_{min}$ is the objective function which accounts for the operating cost of providing a utility for the process requirements and the annualised cost of the equipment for heat exchange. The objective is to minimise the total cost of the operation $Cost_{tot}$ This is formulated as Equation \ref{eq:DTminobj}

\begin{equation}\label{eq:DTminobj}
min \, Cost_{tot}= Cost_{op}+Cost_{inv}
\end{equation}

Where the operating cost, $Cost_{op}$, is calculated as the cost for providing an external utility to accomplish the process requirements instead of using heat integration. This is expressed mathematically as shown in Equation \ref{eq:DTminopcost}