
\section{Heat integration}
% (4p) François + Ivan
Integration of heat in industrial processes has been practiced historically based on engineering intuition and improvements on existing processes. Linnhoff proposed a systematic methodology for heat integration in 1983 and referred to the technique as 'Pinch analysis' \cite{linnhoff_1983_pinch}. This development formed the basis of heat integration techniques in industry since, leading to improved efficiency in the design of modern processes and also providing a basis for targeting of retrofit projects in existing industrial sites. The pinch analysis principles are based on separating a process or site into the streams which require heat, cold streams, and those which have excess heat or requires cooling, hot streams. The aggregate hot and cold streams can be represented on a plot of temperature against heat load for the stream to show the total cooling and heating needs of a process and are named composite curves as they represent the total heating and cooling needs of all streams within a process. Designing an integrated and heat-efficient process suggests that the hot process streams and cold process streams can exchange heat directly, or through an intermediate fluid, in order to maximise the recovery of heat within the process and therefore reduce the external utility load. This can be visualised graphically by adjusting the horizontal positioning of the composite curves until they are separated by a minimum temperature (vertical line) of $\Delta T_{min}$, discussed in section \ref{sec:DTmin}. This separation expresses the minimum temperature that is required to ensure a sufficient driving force for heat to flow from the hot to cold stream and is determined by an optimisation accounting for the capital cost of equipment for heat exchange and the cost of providing an external utility to accomplish the thermodynamic requirement. The formulation of the $\Delta T_{min}$ problem is a linear programming problem and can be formulated as shown in Equations \ref{eq:DTminobj} - \ref{eq:DTminqex} of section \ref{sec:DTmin}. Section \ref{sec:ccs} describes the construction and meaning of the composite curves while Section \ref{sec:cascade} carries the analysis further into recovering and reusing heat with the heat cascade formulation.

\subsection{Determining $\Delta T_{min}$}\label{DTmin}
The first equation in determining the appropriate $\Delta T_{min}$ is the objective function which accounts for the operating cost of providing a utility for the process requirements and the annualised cost of the equipment for heat exchange. The objective is to minimise the total cost of the operation $Cost_{tot}$ This is formulated as Equation \ref{eq:DTminobj}

\begin{equation}\label{eq:DTminobj}
min \, Cost_{tot}= Cost_{op}+Cost_{inv}
\end{equation}

Where the operating cost, $Cost_{op}$, is calculated as the cost for providing an external utility to accomplish the process requirements instead of using heat integration. From the objective function, it can be seen that selecting $\Delta T_{min}$ guarantees that suggested heat exchange options are economically feasible by definition as the investment and operating costs are both considered. The economic expression can be further expressed mathematically as shown in Equation \ref{eq:DTminopcost}.
\begin{equation}\label{eq:DTminopcost}
Cost_{op}=\left(Cost_{CU} (\dot Q_{hot}-\dot Q_{ex})+ Cost_{HU} (\dot Q_{cold}-\dot Q_{ex})\right) \cdot Time_{op}
\end{equation}

Where the heat loads $\dot Q_{hot},\, \dot Q_{cold}, \, \text{and }\dot Q_{ex}$ are the heat load from the hot stream, cold stream and exchanged between the hot and cold streams, respectively. $Cost_{CU} \, \text{and } Cost_{HU}$ are the specific costs of the cold and hot utilities, respectively, and $Time_{op}$ is the annual operating time. The investment cost, $Cost_{inv}$, to realize $\dot Q_{ex}$ is related through the heat exchange area $A_{ex}$ according to:

\begin{equation}\label{eq:DTmininvcost}
Cost_{inv}=F_a a_{ex}(A_{ex})^{b_{ex}}
\end{equation}

Where $a_{ex}$ and $b_{ex}$ are economic parameters found in literature or by conducting an analysis of current heat exchanger area costs. Typically the fixed cost of the exchanger is small compared to the area-dependent cost and it is not considered in this formulation. The annualisation factor, $F_a$ is based on the lifetime of the project in years, $n$, and the interest rate, $i$, associated with the investment. These are significant economic parameters and are typically decided at the level of the plant operation, project assessment team or other economic roles within an industry. The factor can be calculated for any combination of these parameters as shown in \ref{eq:DTminamort} and can play a major role in determining the $\Delta T_{min}$; thus, these economic parameters must be known before conducting the analysis to ensure that the solutions are indeed economically feasible.

\begin{equation}\label{eq:DTminamort}
F_a=\left(\frac{i(1+i)^n}{(1+i)^n-1}\right)
\end{equation}

The total area required for heat exchange, $A_{ex}$, is based on heat transfer principles and described in Equation \ref{eq:DTminarea}.

\begin{equation}\label{eq:DTminarea}
A_{ex}=\frac{\dot Q_{ex}}{U_{ex}\Delta T_{lm}}
\end{equation}

Where the heat exchange coefficient, $U_{ex}$ must be calculated for the specific case of the heat exchange depending on the fluids involved. The term $\Delta T_{lm}$ is the logarithmic mean of the temperature difference between the fluids in the in exchanger. It is related to the inlet and outlet temperatures of the hot and cold fluids according to the mathematical relation shown by Equation \ref{eq:DTmintlm}.

\begin{equation}\label{eq:DTmintlm}
\Delta T_{lm}= \frac{(T_{hot,in}-T_{cold,out})-(T_{hot,out}-T_{cold,in})}{\ln\left(\frac{(T_{hot,in}-T_{cold,out})}{(T_{hot,out}-T_{cold,in})}\right)}
\end{equation}

The transfer of energy required from a hot or cold utility can be defined using Equations \ref{eq:DTminq1} and \ref{eq:DTminq2}, respectively. These two equations are descriptive of the load difference required to heat a cold stream or cool a hot stream after considering the exchange of heat between the hot and cold stream. These are directly related to the operating cost portion of the objective function.

\begin{equation}\label{eq:DTminq1}
\dot Q^+=\dot Q_{cold}-\dot Q_{ex}
\end{equation}

\begin{equation}\label{eq:DTminq2}
\dot Q^-=\dot Q_{hot}-\dot Q_{ex}
\end{equation}

Which means that Equation \ref{eq:DTminopcost} can be rewritten in a condensed format including these concise definitions as shown by Equation \ref{eq:DTminopcost2}

\begin{equation}\label{eq:DTminopcost2}
Cost_{op}=\left(Cost_{CU} \dot Q^-+ Cost_{HU} \dot Q^+\right) \cdot Time_{op}
\end{equation}

The temperatures of the hot and cold streams must then be defined relative to the amount of heat exchange occurring between the streams. The hot and cold outlet temperatures from the exchange are calculated in Equations \ref{DTmintc} and \ref{eq:DTminth} which, of course, are dependent on $\dot Q_{ex}$ and thus on $\Delta T_{min}$.

\begin{equation}\label{eq:DTmintc}
T_{cold,out}=\frac{\dot Q_{ex}}{\dot M_{cold}C_{p_{cold}}}
\end{equation}

\begin{equation}\label{eq:DTminth}
T_{hot,out}=T_{cold,in}+\Delta T_{min}
\end{equation}

Finally, the dependent variable $\dot Q_{ex}$ is defined in terms of the stream temperatures and the $\Delta T_{min}$ in Equation \ref{eq:DTminqex}.

\begin{equation}\label{eq:DTminqex}
\dot Q_{ex} = \dot M_{hot}C_{p_{hot}}(T_{hot,in}-(T_{cold,in}+\Delta T_{min}))
\end{equation}

\subsection{Composite and grand composite curves}\label{sec:ccs}

\subsection{Heat recovery and the heat cascade}\label{sec:cascade}