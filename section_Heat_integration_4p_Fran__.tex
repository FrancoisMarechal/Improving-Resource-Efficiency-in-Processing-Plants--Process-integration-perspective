
\section{Heat integration}
% (4p) François + Ivan
Integration of heat in industrial processes has been practiced historically based on engineering intuition and improvements on existing processes. Linnhoff proposed a systematic methodology for heat integration in 1983 and referred to the technique as 'Pinch analysis'. This development formed the basis of heat integration techniques in industry since, leading to improved efficiency in the design of modern processes and also providing a basis for targeting of retrofit projects in existing industrial sites. The pinch analysis principles are based on separating a process or site into the streams which require heat, cold streams, and those which have excess heat or requires cooling, hot streams. The aggregate hot and cold streams can be represented on a plot of temperature against heat load for the stream to show the total cooling and heating needs of a process. Designing an integrated and heat-efficient process suggests that the hot process streams and cold process streams can exchange heat directly, or through an intermediate fluid, in order to maximise the recovery of heat within the process and therefore reduce the external heating load.