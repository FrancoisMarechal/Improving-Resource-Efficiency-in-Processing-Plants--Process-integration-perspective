\section{Retrofit of HEN: Targeting}
In this part the previously presented concepts of the multiple representation of the same energy requirement and the MILP model for optimal utility integration is extended to target the complete retrofit problem. Both the single objective and multi-objective optimization have been comprehensively inspected.


\subsection{Defining Heating/Cooling requirement to target retrofit problem}
\label{subsec6:heattransinterface}

The identified energy requirement of one plant is categorized into two main groups: The hidden potential of heat transfer and the existing heat transfer units. The hidden potential of heat transfer relates to the hot or cold sources of energy that are not currently considered as a potential of heat transfer, such as the streams which leave (or enter to) the boundary of the system with a higher or lower temperature than the reference. The existing heat transfer units are categorized into six types: Process-Process heat exchanger (P-P), Process-Utility heat exchanger (P-U), Utility-Utility heat exchanger (U-U), Heat transfer intermediate network (Loop), Non-Isothermal mixing and Internal heat transfer in reactors. 

In an existing plant, several heat transfer units might fulfill the same heating/cooling requirement. This approach offers the possibility to define the retrofit process integration by introducing each of the heat transfer units of one energy requirement with particular representation. \cref{fig6:triple2} shows an example of a process heat requirement with three heat transfer units. The heat demand of the process stream is satisfied with the hot water, which receives the heat from the steam. The steam condensate in its turn receives the same amount of heat from combusted gases in boiler to become steam. The hot water and steam loops are considered as technology representation, combusted gases as utility and process streams as the thermodynamic ones. 

In summary, one energy requirement can be presented by several heat transfer units (P-P, P-U or U-U), that all are identical with the energy quantity but each differentiates through its exergy content. They can also be defined with one particular representation (Utility, Technology or Thermodynamic) as introduced in \cref{chap2}.

%\begin{figure*}[!ht]
%\centering
%\includegraphics[width=125mm]{images/ch6/triplerep2.pdf} 
%\caption{Multiple representation of the same energy requirement} \vspace*{-5mm}
%\label{fig6:triple2}
%\end{figure*}

\subsection{Optimization with unique representation}
\label{sec:retrofitoptunique}
In the first optimization step, the heat cascade is solved and the minimum energy requirement (MER) is calculated by defining the requirements with the most available detailed representation. The MILP model with only one representation for each energy requirement to minimize the operating cost is identical to the developments explained in \cref{Sec3:EnergyConvOpt} which we do not reproduce here for the sake of conciseness. 

\subsection{Optimization with multiple representation}
\label{sec6:multipleMILP}

To identify the required modification in the retrofit problem, all available heat transfer units on the site have to be identified and introduced to the MILP model. The existing heat exchangers in the new system either remain unchanged and are accepted as they are or they have to be removed/modified as the penalizing ones, depending on their effects on the energy consumption. The penalizing heat exchangers are the ones responsible for an increase in energy consumption by either Cross-Pinch heat transfer or inappropriate heat exchange in heat source or heat sink. The Cross-Pinch exchangers are those who transfer heat across the pinch by the P-P exchange, use the hot utility below the pinch or use the cold utility above the pinch. These rules are valid for both process and utility pinch points. All of the heat transfer categories of \cref{subsec:ExtHexs} can be summarized into P-P, P-U and U-U heat exchangers. The P-P heat exchangers are a part of the heat recovery potential calculated by heat integration. If an existing P-P exchanger does not cross-pinch and is well placed in the system, accepting this exchanger and removing its streams from the system would not affect the operating cost. In case of a penalizing existing P-P exchanger, the imposed penalty can be evaluated as the difference between the operating cost of the system, with and without streams of this exchanger. Regarding the P-U heat exchangers, if they are part of a loop, with intermediate heat transfer fluid carrying heat between process heat sink and source (see \cref{fig6:PUUUexchangers}.a), they are considered and analyzed similarly to the P-P heat exchangers. Otherwise, accepting a well placed P-U exchanger would reduce the operating cost as much as the cost of the utility used in the P-U exchanger. However, in order to guarantee that the utility stream in the exchanger is not different from the optimum choice of the optimization, the exergy losses in the exchanger should also be evaluated. If a U-U exchanger is part of a loop, with intermediate heat transfer fluid carrying heat between process and utility streams (see \cref{fig6:PUUUexchangers}.b), they are considered and analyzed similarly to the P-U heat exchangers. But if a U-U exchanger is not part of a loop nor for a specific process reason, it is a cross-pinch exchanger. 

 
 \begin{figure}[!ht]
     \centering
     \begin{tabular}{cc}
        \subfloat[P-U heat exchnagers]{\includegraphics[width=90mm]{images/ch6/P-U-LOOP.pdf}} \\
        \subfloat[U-U heat exchnagers]{\includegraphics[width=75mm]{images/ch6/U-U-LOOP.pdf}}
     \end{tabular}
    \caption{Existing heat exchangers in a loop}
     \label{fig6:PUUUexchangers}
 \end{figure}
 
 
This analysis can be performed for all the existing heat exchangers by performing a remaining problem analysis. However, the decision made concerning one exchanger will affect the result of the analysis for the rest. So the sequence of the modifications becomes important and we do not necessarily know which sequence leads to the optimum solution in terms of the target. Additionally, among the available heat transfer units of each energy requirement, it is not clear which heat transfer units of each energy requirement have to be modified and thus the optimum retrofit solution.

In order to deal with this problem, all available heat transfer units of each energy requirement are introduced to the optimization problem. Each interface is defined by one energy requirement representation. The process requirement is defined by the thermodynamic representation and the current utility interface is defined by the utility representation. All interfaces interacting between process and utility will be defined by the technology representation. For each requirement, the optimization problem selects the optimum interface with respect to the target. To guarantee the presence of only one interface for each energy requirement at once, a new constraint (\cref{EQ6:constrain}) is added to the MILP formulation. This constraint has been defined with the integer variable that represents the existence of one unit ($y_{u}$). In the original MILP formulation $y_{u}$ has a constant value of 1 for all process units. A new associated integer variable to each representation of a process unit is defined: in case the process unit $u$ takes the representation $r$ ($y_{u,r}=1$) and ($y_{u,r}=0$) otherwise.

\begin{equation}
\sum_{r=1}^{nr_{u}} y_{u,r}=1  \quad  \quad \forall u=1,...,nu_{p}  \quad \quad  y_{u,r}\in 
\left\{0,1 \right\}  \quad\quad 
\label{EQ6:constrain}
\end{equation} 

 where $nr_{u}$ is the number of representations of the process unit $u$ and $nu_{p}$ is the number of utility-consumer process units with more than one representation. Thermodynamic representation is chosen when the energy requirement is part of the energy saving target and the heat transfer unit has to be modified. Choosing the technology representation means that only heat supply through heat transfer intermediate unit can be modified, but not the process unit itself. When the energy requirement is defined by utility representation, neither the process, nor the utility supply can be modified.  Therefore, including one stream with utility representation means there is no need to modify any of the current heat transfer units of this requirement. 

In addition to the process units that exchange with utility or technology interfaces, some process units are introduced to the problem that belong to the hidden potential of heat transfer, the existing P-P heat exchangers or the non-isothermal mixers categories. These units have only the possibility to be represented by thermodynamic requirement and to evaluate if they are beneficial to be included in the optimum retrofit target, several iterations would be required. To deal with this problem, another constraint (\cref{EQ6:constrain2}) is added to the MILP model for each of those units. This constraint allows to vary the integer variable that represents the existence of the unit ($y_{u}$) from being a part of the model ($y_{u}=1$) to the elimination ($y_{u}=0$).

\begin{equation}
  y_{u} = y_{u}^{\star }  \quad  \quad \forall u=1,...,nu_{p}^{\star }  \quad \quad  y_{u}^{\star }\in 
  \left\{0,1 \right\}
\label{EQ6:constrain2}
\end{equation}

where $nu_{p}^{\star }$ is the number of process units who belong to the hidden potential of heat transfer, existing P-P heat exchangers or non-isothermal mixers categories and are accordingly represented only through the thermodynamic representation. The binary variable $y_{u}^{\star } $ is defined $(y_{u}^{\star }=1)$ when the unit $u$ is included in the model and has to be modified or $(y_{u}^{\star }=0)$ when unit $u$ is eliminated from the model and is accepted as it is. 

There might be some engineering constraints and physical limits concerning the modification or adding the new heat exchangers. These constraints have also to be added to the optimization problem at the targeting level. The constraints related to the limit on the exchange at certain temperatures are guaranteed by introducing new representation for the limited unit. The forbidden matches can also be considered by using the extension of the MILP formulation for restricted matched proposed by \citet{Becker2012104}. 
 
\subsection{Multi-Objective optimization with multiple representation}
\label{Sec6:MultiObjOptz}

Apart from securing the optimum solution with respect to the operating cost, the systematic alteration of the heat transfer unit of the energy requirements from one representation to another also leads to a new target retrofit solution. Switching among representations, however, involves the expenses to have the heat exchangers modified. In order to design the proposed retrofit solution, the cost of each modification has to be evaluated. Hence, earlier at the level of targeting, it would be helpful to have an estimation of the units which have to be changed for the optimum retrofit design with respect to both operating and investment cost. For this purpose, a multi-objective optimization with evolutionary algorithm \cite{Molyneaux2010751} is performed by considering both operating and investment cost as targets which would give the optimal Pareto-frontier of solutions. Pareto analysis is practical in our case to identify and characterize interesting regions of the solution space of the trade-off between objectives instead of a single optimal point. Generating a range of $good$ $solutions$ will provide the decision maker options to choose a qualitative solution among the regions returned by the algorithm. Notice that each of these options is identical in terms of the energy balance between the hot and the cold requirement in spite of their different operating and investment costs. The investment cost of the heat exchanger modification is estimated by associating the area of the heat exchanger to its hot and cold stream (\cref{EQ6:equ15}). In order to forward the optimization towards the solutions with less required modification, a systematic weight factor is assigned to each representation to promote the utilization of existing facilities (\cref{EQ6:equ22}). In addition, a pseudo thermal exergy destruction is calculated as an indicator to compare different solutions. 

\subsection{Caste study III: Industrial cluster integration and optimization}

The proposed approach has been applied on a large-scale industrial site with three process systems\footnote{Called as A, B and C}. The total site has 141 process units, including 172 streams. 

As the first step, the heat cascade is solved and the minimum energy requirement (MER) is calculated by defining the requirements with the most available detailed representation. \cref{fig6:mer} shows that the heat recovery potential can be enhanced from 28\% in the current situation to 52\% of the total heat requirement by integration. Using the MILP formulation proposed by \citet{marechal1998process} the heat cascade is solved, suitable energy conversion units are integrated and combined heat and power production is maximized by minimizing the operating cost. This method selects the equipment in the superstructure and determines their optimal operating flow rate in the integrated system. The price for different forms of energy influences the choice of the appropriate technology and its level of usage in the integrated solution. 

\begin{figure*}[!ht]
\vspace{2mm}
\centering
\includegraphics[width=150mm]{images/ch6/MER.pdf} 
\caption{Site Composite and Grand Composite Curves with unique representation of energy requirement}
\label{fig6:mer}
\end{figure*}
\vspace{5mm}


\cref{fig6:MILP1st} shows the result of the site utility integration and optimization, for two scenarios. The scenario I include boiler, steam network with steam turbine and refrigeration cycle (\cref{fig6:MILP1st}a), and scenario II has a cogeneration system with gas turbine in addition to them (\cref{fig6:MILP1st}b). Here, the ratio of gas price to the electricity is supposed to be 0.4. Moreover, the capacity of the cogeneration unit is optimized to produce equivalent heat to the MER of the system. As a result, the electricity that is produced by the cogeneration unit, comes only from the excess exergy content of the necessary required gas that satisfies MER. The finalized solutions are then evaluated based on the operating and HEN investment cost and are also compared to the current situation of the plant. The summary of the analysis is presented in \cref{tab6:one}. The reported relative balanced electricity is the ratio of the system balanced electricity (electricity production minus consumption) over the equivalent  kW of current natural gas consumption. The presented thermal exergy destruction is also the relative value with respect to the same term. The relative annualized investment cost is the ratio of the investment cost for each scenario, over the current operating cost. The extended MILP formulation is solved for scenario I and II with all available heat transfer units of each energy requirement to find the minimum operating cost (\cref{fig6:MILP2nd}). 


 \begin{figure}[!ht]
 \vspace{2mm}
 \begin{center}
 \includegraphics[width=70mm]{images/ch6/MILP1stCH3.pdf}\quad
 \includegraphics[width=70mm]{images/ch6/MILP1stCH4.pdf} 
 \caption{Energy conversion units integration and optimization with unique representation for scenario I (a) and II (b)} 
 \label{fig6:MILP1st}
 \end{center}
 \end{figure}

\begin{table}[!ht]
  \caption{Summary of the optimized site utility integration}
     \label{tab6:one} 
  \centering
\resizebox{14.5cm}{!} {
\begin{tabular} {m{7.6cm} >{\centering\arraybackslash}  m{1.3cm} >{\centering\arraybackslash} m{2.1cm} >{\centering\arraybackslash}  m{2.2cm} >{\centering\arraybackslash}  m{2.1cm} >{\centering\arraybackslash} m{2.2cm} >{\centering}  *{6}{c}}
\noalign{\hrule height 0.06cm}
\noalign{\smallskip}
 {} & \textbf{Current site }& \textbf{Scenario I unique.rep} & \textbf{Scenario II unique.rep} & \textbf{Scenario I mult.rep} & \textbf{Scenario II mult.rep}\\ [0.1cm]
\hline \vspace*{2mm}
\textbf{Relative Natural Gas [\%]}  & \vspace*{2mm} 100 & \vspace*{2mm} 65 & \vspace*{2mm} 103 & \vspace*{2mm} 64 & \vspace*{2mm} 103  \\ 
\textbf{Relative balanced electricity [\%]}  & 0& 12 & 35& 12 & 35\\ 
\textbf{Relative thermal exergy destruction [\%]}  & - & 21 & 15 & 21 & 14\\ 
\textbf{Energy saving [\%]}  & 0  & 43 & 45 & 44 & 45 \\
\textbf{Relative Operating cost [\%]}  & 100  & 49 & 33 & 48 & 32  \\ 
\textbf{Relative Annualized Investment [\%]}  & 0  & 4.4 & 2.6 & 4.1 & 2.6  \\
\textbf{Relative Total cost [\%]}  & 100 & 54 & 36 & 52 & 35\\ 
\noalign{\hrule height 0.06cm} 
\end{tabular}}
\vspace{6mm}
        \end{table}
 
 The summary of the result is presented in \cref{tab6:one}. The value of reported thermal exergy destruction is the area between process and utility curves in the Carnot representation (\cref{fig6:MILP2nd}) that specifies the exergy losses in the heat transfer system \cite{marechal2005combined}. The multi-objective optimization with four objective functions of the operating cost, the investment cost, the priority and the pseudo thermal exergy destruction has been performed for the scenario I by varying the streams over different representations. 133 out of 141 units of the site, can be modified and switched from one interface to another one. The 8 others were mandatory to remain unchanged. In the existing facilities of this site, in addition to the thermodynamic and utility interfaces, there exist 9 different technology interfaces. 

 \begin{figure}[!ht]
 \begin{center}
 \includegraphics[width=70mm]{images/ch6/MILP2ndCH3.pdf} \quad
 \includegraphics[width=70mm]{images/ch6/MILP2ndCH4.pdf} \
 \caption{Energy conversion units integration and optimization with multiple representation} \vspace*{-5mm}
 \label{fig6:MILP2nd}
 \end{center}
 \end{figure}
 

The Pareto analysis has been performed by selecting the optimal Pareto-frontier of all four objectives and is displayed in the 3D format with respect to the investment cost, the operating cost and the number of modifications in \cref{fig6:multiobj}a. The projection of this frontier on the respective planes gives the 2D trade-off which has been reproduced in \cref{fig6:multiobj}b and c. \cref{fig6:multiobj}b illustrates the trade-off between the operating cost and the investment cost. The point 1 is the optimum solution to the MILP optimization problem where all requirements are exclusively supposed to be of the detailed representation type and the point 2 is the optimum solution for a system allowing for the multiple representation of energy requirements with only the operating cost as objective function. Each point of this region presents a retrofit solution which includes the information concerning the heat transfer unit of each energy requirement that need to be modified. The Pareto curve analysis shows that among the solutions with the same operating cost, one can be found by the help of the multiple representation approach, which requires less investment cost compared to the results obtained by the unique representation approach. The selected solution of the above optimization problem is then the target for a reduced HEN design problem which is limited to the number of units which would require modification. \cref{fig6:multiobj}b plots the number of units that requires modification with respect to the operating cost. This curve allows the engineer to determine which and how many units are required to be modified for any target solution.
 
  \begin{figure}[!ht]
  \vspace{2mm}
  \centering
  \begin{tabular}{cc}
  \multicolumn{2}{c}{ 
  \includegraphics[width=80mm]{images/ch6/3d.pdf}
  } \\
  \includegraphics[height=58mm]{images/ch6/multiobja1.pdf} &
           \includegraphics[height=58mm]{images/ch6/multiobjb.pdf}
 \end{tabular}
         \caption{Pareto analysis of scenario I for multiple representation of each energy requirement}    \label{fig6:multiobj} \vspace*{-2mm}
         \end{figure}
         \vspace{3mm}
         
In order to compare the value of the four objectives for each solution, the Pareto trade-off between operating cost and all the three other objectives is shown in \cref{fig6:multall}. Point A shows an example of a $good$ solution that requires 42 process unit modifications, nearly 2\% investment cost and 51\% operating cost. Although this target solution saves 3\% less than the MILP optimum Point 2, its implied number of modifications are considerably less than the 108 required modifications for the optimum point 2. Supposing that the decision maker chooses the Point A as the solution scenario, up to 91 units from the total of 133 can use the existing thermodynamic interfaces. In point A, 42 units are represented by thermodynamic, 10 by technological and 47 by utility requirement. Also 34 units out of 42 units from the P-P heat exchangers, non-isothermal mixers and hidden potential categories are eliminated from the target solution. This is the first significant information for the decision maker, that 34 units have been fully accepted as they are. Moreover, the elimination of these units reduces the number of streams from 172 to 114 which has a considerable impact on the HEN design step. For the 10 units represented by the technology requirement, the heat exchanger between process and technology flows may remain without any modification. Instead, if there is a need, it is the technology flow that will participate in the heat exchange. For the 47 units that appeared with their utility representation, neither of their thermodynamic and technology interfaces would require modification. 

    \begin{figure}[!ht]
    \vspace{2mm}
       \begin{center}
       \includegraphics[width=100mm]{images/ch6/multall.pdf} 
       \caption{Pareto curve of investment cost, number of modification and pseudo thermal exergy destruction vs operating cost} \vspace*{-5mm}
       \label{fig6:multall}
       \end{center}
       \vspace{4mm}
       \end{figure}   

The utility and technology representation streams can even hint at more practical information. We define $T_{hot}^{max}$ as the highest corrected temperature of the hot process streams and $T_{cold}^{min}$ as the coldest corrected temperature of the cold process stream (\cref{fig6:Tmaxmin}). If the corrected temperature of a cold stream is higher than $T_{hot}^{max}$, this stream should exchange with hot utility or heat transfer unit. It is not thus a part of the heat recovery and can be eliminated from the heat recovery network retrofit problem. The analogous argument is valid for the hot stream with higher corrected temperature than $T_{cold}^{min}$. Obviously, probability of appearance of such a stream, increases by having the possibility to switch from the thermodynamic representation to the technology or utility ones. 
        
                   \begin{figure}[!ht]
                     \vspace*{2mm}
                    \begin{center}
                    \includegraphics[width=80mm]{images/ch6/tmaxmin.pdf} \vspace*{0mm}
                    \caption{Analysis of streams participating in heat recovery using extreme temperatures
                    } \vspace*{0mm}
                    \label{fig6:Tmaxmin}
                    \end{center}
                    \end{figure} 
                    
The above mentioned analysis has been performed for the point A and the result is summarized in \cref{tab6:four}. It is shown that out of the 21 streams that are beyond the extreme temperatures, 20 of them are presented by utility or technology requirements and only 1 stream has thermodynamic representation. This result again highlights the benefit of having multiple heat transfer units for the same energy requirement rather than employing a unique one.

\begin{table}[H] 
 \vspace*{2mm}
  \caption{Analysis of streams with extreme temperature for point A}  \vspace*{-2mm}
     \label{tab6:four} 
  \begin{center}
\resizebox{14.7cm}{!} {
\begin{tabular} {m{7.7 cm} >{\centering\arraybackslash}  m{3.4cm} >{\centering\arraybackslash} m{1.6cm} >{\centering\arraybackslash}  m{1.5cm} >{\centering\arraybackslash}  m{2.2cm } >{\centering\arraybackslash}  m{3.2cm } >{\centering} *{6}{c}  }
\noalign{\hrule height 0.06cm}
\noalign{\smallskip}
 \textbf{Category} & \textbf{Extreme Temperature $[^{\circ}\mathrm{C}]$} &  \textbf{Number} & \textbf{Utility} & \textbf{Technology} & \textbf{Thermodynamic} \\ [0.1cm]
\hline
\noalign{\vskip 1mm} 
Cold streams with higher temperature &\textbf{ $T_{hot}^{max}=196$}  &  9 & 7 & 1 & 1\\ [0.2cm]
Hot streams with lower temperature  &\textbf{$T_{cold}^{min}=29$}  &  12 & 12 & 0 & 0\\ [0.2cm]
Hot streams with partially lower temperature & \textbf{$T_{cold}^{min}=29$} &  21 & 18 & 0 & 3 \\ [0.1cm]
\noalign{\vskip 1mm} 
\noalign{\hrule height 0.06cm} 
\end{tabular}}
\vspace*{0mm}
       \end{center}
        \end{table}

 Finally, these streams also can be subtracted from the stream list for HEN and reduce the 114 streams to 93 (\cref{fig6:HENanalysis}). Notice that the elimination of these streams does not affect the target objectives. There are some streams which are partially beyond the extreme temperatures (by either of the entrance or outlet temperature). In case more simplification is required, a part of these streams might also be eliminated from the model.

        
           \begin{figure}[!ht!]
             \vspace*{2mm}
            \begin{center}
            \includegraphics[width=40mm]{images/ch6/HENanalysis.pdf} \vspace*{0mm}
            \caption{Procedure of streams reduction for the HEN design model of point A} \vspace*{-5mm}
            \label{fig6:HENanalysis}
            \end{center}
            \end{figure} 
