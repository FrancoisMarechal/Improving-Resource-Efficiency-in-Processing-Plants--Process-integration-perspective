the identified energy requirement of one plant is categorized into two main groups: The hidden potential of heat transfer and the existing heat transfer units. \section{Retrofit of HEN: Targeting}

\subsection{Defining Heating/Cooling requirement to target retrofit problem}
\label{subsec6:heattransinterface}
The identified energy requirement of one plant is categorized into two main groups: The hidden potential of heat transfer and the existing heat transfer units. The hidden potential of heat transfer relates to the hot or cold sources of energy that are not currently considered as a potential of heat transfer, such as the streams which leave (or enter to) the boundary of the system with a higher or lower temperature than the reference. The existing heat transfer units are categorized into six types: Process-Process heat exchanger (P-P), Process-Utility heat exchanger (P-U), Utility-Utility heat exchanger (U-U), Heat transfer intermediate network (Loop), Non-Isothermal mixing and Internal heat transfer in reactors. 

In an existing plant, several heat transfer units might fulfill the same heating/cooling requirement. This approach offers the possibility to define the retrofit process integration by introducing each of the heat transfer units of one energy requirement with particular representation. \cref{fig6:triple2} shows an example of a process heat requirement with three heat transfer units. The heat demand of the process stream is satisfied with the hot water, which receives the heat from the steam. The steam condensate in its turn receives the same amount of heat from combusted gases in boiler to become steam. The hot water and steam loops are considered as technology representation, combusted gases as utility and process streams as the thermodynamic ones. 

In summary, one energy requirement can be presented by several heat transfer units (P-P, P-U or U-U), that all are identical with the energy quantity but each differentiates through its exergy content. They can also be defined with one particular representation (Utility, Technology or Thermodynamic) as introduced in \cref{chap2}.

%\begin{figure*}[!ht]
%\centering
%\includegraphics[width=125mm]{images/ch6/triplerep2.pdf} 
%\caption{Multiple representation of the same energy requirement} \vspace*{-5mm}
%\label{fig6:triple2}
%\end{figure*}

\subsection{Caste study III: Industrial cluster integration and optimization}