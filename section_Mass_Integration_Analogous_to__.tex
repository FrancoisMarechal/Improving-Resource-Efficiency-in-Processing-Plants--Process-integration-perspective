\section{Mass Integration}

Analogous to energy integration, it is also conceivable to integrate material streams of a process. Mass as easily converted between forms as is energy which necessitates a slightly altered consideration for improving process integration. Common examples of this practice are for commonly used chemicals or utilities such as hydrogen, water, $CO_2$ or others. The analog between material integration and heat integration is that the flowrate is considered as the analog for heat load and the quality of the material as the analog for temperature. 

\subsection{Hydrogen Network Integration}
As mentioned, hydrogen can be considered as an analog to heat and possesses the following attributes:
\begin{itemize}
\item Is commonly transported through a network of pipes
\item Has varying purity
\item Can be produced and consumed by many different technologies\\units
\item Is produced and consumed at various purities and flowrates throughout industrial plants
\item Quality can be altered by using purification technologies
\end{itemize}

This makes hydrogen an ideal candidate to be considered in the same way as heat. A formulation of the problem for designing a hydrogen network was published by Girardin et al. \cite{girardin_methodology_2006}. This formulation is constructed in a similar way to the heat cascade problem and the objective function is again to minimize the total annual cost which is a construct of the annualised investment cost and the operating cost.

\subsection{Hydrogen Network Design Optimization}
The formulation of the hydrogen network integration problem is explained in this section. Such a methodology could also be applied to other material networks such as $CO_2$ or water, given that they have measurable purities which would be improved or degraded by use in different technologies, are contained within a network structure and have various production and consumption throughout a facility.  